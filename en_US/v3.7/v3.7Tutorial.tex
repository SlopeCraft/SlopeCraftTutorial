%\documentclass[UTF8]{ctexart}
\documentclass{article}
\usepackage{geometry}
\usepackage{fancyhdr}
\usepackage{verbatim}
\usepackage{enumerate}
\usepackage{graphicx}
\usepackage{subfigure}
\usepackage[colorlinks,linkcolor=blue]{hyperref}
\usepackage{listings}
\usepackage{fontspec}
\usepackage{soul}
\setmonofont{Consolas}
\usepackage{color}
\usepackage{xcolor}
\graphicspath{{./Images/}}

\geometry{papersize={210mm,297mm}}
\geometry{left=2.5cm,right=2.5cm,top=2.5cm,bottom=2.5cm}

\setlength{\headheight}{13pt}

\title{SlopeCraft v3.7 UserGuide}
\author{TokiNoBug}
\date{\today}

\begin{comment}
\pagestyle{fancy}

\lhead{\author}
\chead{\title}
\rhead{\date}

\lfoot{}
\cfoot{\thepage}
\rfoot{}

\renewcommand{\headrulewidth}{0.4pt}
\renewcommand{\headwidth}{\textwidth}
\renewcommand{\footrulewidth}{0pt}

\end{comment}

\begin{document}
    \maketitle
    %%%%%%%%%%%%%%%%%%%%%%%
    % Page1
    %%%%%%%%%%%%%%%%%%%%%%%
    Welcome to use SlopeCraft v3.7! You will learn all functions of SlopeCraft in this document.

    This tutorial is not a fool-type one, only those must be introduced are introduced and those must be underlined are underlined. If any question, read a previous one.

    %%%%%%%%%%%%%%%%%%%%%%%
    % Page2
    %%%%%%%%%%%%%%%%%%%%%%%
    %\pagebreak
    \section{Update Summary}
    \paragraph{New Contents}

    \textbf{imageCutter}——A lite image preprocessor that can scale and cut images.

    \paragraph{New Functions}
    \begin{enumerate}
        \item Update to MC1.18
        \item New image converting algorithm to reserve edge better : AiCvter (genetic algorithm actually)
    \end{enumerate}

    \begin{comment}  
    \paragraph{Bug Fixed}
    \begin{enumerate}
        \item Fixed the bug that may colors can't be disabled.
    \end{enumerate}
    \end{comment}

    \paragraph{Optimization and Other Changes}
    \begin{enumerate}
        \item \textbf{SlopeCraftL3.dll}——The kernel dll has another C API version, which might be easier to be called by other languages.
    \end{enumerate}

    \pagebreak
    %%%%%%%%%%%%%%%%%%%%%%%
    % Page3
    %%%%%%%%%%%%%%%%%%%%%%%
    \section{Junior Tutorial}

    \subsection{Preprocess Image}
    Minecraft maps has a resolution which is multiples of 128 pixels, so it's really recommended to resize your image sothat your image can fully fill a single map or more.

    The image preprocessor, \textbf{imageCutter.exe} can resize your image in 3 modes and using 2 transformation methods. However image scaling is far more complex than it seems to be, there're varities of professional softwares which can scale much better than my naive imageCutter, like PhotoShop.

    If you want to crop your image, \textbf{imageCutter.exe} isn't able to do that. You should do that with a more expert software.

    Last but not least, the most improtant function of \textbf{imageCutter.exe} is to cut image into several 128 by 128 blocks. There have been some users hoping to cut their schematics into 128 by 128 parts, so the image cutting function together with batch operation will satisfy them ---- images are obviously more easier to be cut than schematics.

    \subsection{Import Image}
    From the previous tutorial to now, the greatest change is still a new button \textbf{Settings}, which is used to process transparent pixels better.
 
    Click \textbf{Settings} and you will see a subwindow as figure \ref*{SetTPS}.

    \begin{figure}[htbp]
        \centering
        \includegraphics[width=15cm]{Img1_TPS.png}
        \caption{Subwindow to set transparence processing strategy}
        \label{SetTPS}
    \end{figure}
    
    \textbf{Note: If you want to import a image with transparency under custom transparency processing strategy, you must set the strategy first! Otherwise the image will only be preprocessed by the default strategy! If you set the strategy after having your image imported, load you image again.}

    We have various method to deal with full-transparent pixels(alpha=0) and semitransparent pixels(alpha>0). The former are either replaced with background color or air block, while the latter have 3 choices: to be replaced with background color, to be composed with background color, or to be taken as non-transparent pixel by ignoring its alpha channel. Besides, the background color is also customizable. In default it's light gray.

    Please note that each \textbf{pixel corresponds to a block in map}. Since the width and height are recommended to be any \textbf{multiple of 128 pixels}, you should better cut and scale you image before importing into SlopeCraft.(Recommended but not forced) This preprocessing should be done by user instead of software, so I don't and won't implement such function in SlopeCraft. As a result, there is not any resizing or scaling operation in SlopeCraft, \textbf{each pixel} shall be reflected faithfully.

    For processing multiple images at one time, see \ref{BatchOp}.

    \subsection{Set Map Type}
    The only modification is that a new kind of map (wall maps) is added. You just need to choose the right map type and game version following the description on ui, then turn to the next page.

    The 61th base color: GLOWING\_LICHEN is complemented to the blocklist page. Actually it was already added in Minecraft v1.17 but I left it in SlopeCraft v3.5.1.

    The logic of block list: a disabled checkbox or radiobutton mean locked or unchangeable, for example, basecolor 0(air) must be enabled. SlopeCraft requires that each basecolor must have a block eventhough the basecolor is disabled. Thus, if a basecolor has only one available block, that single radiobutton will fixed.
    
    If a block's version is later than the game version, is will also be disabled. Note that not all block can be used in wall maps, like iron pressure plate and glowing lichen. In such condition, it's not only properties of basecolors and maps themselves but also your selected blocks that restrict your available color count.

    \subsection{Color Convertion}
    
    First, select a converting algorithm, then set dithering or not. Then click \textbf{Convert}, after the convertion is finished, export in the way you want.

    The 6 converting algorithms respectively correspond to 6 different color difference formulations. They care only pixel colors but ignore their position, which made them traditional algorithms. The rest one, \textbf{AiCvter}, uses the above 6 results as seeds and find a better one whose edge map is closer than the orignal image. This new algorithm cares about pixel's space information, thus it's nontraditional.
    
    Among all algorithms, RGB+ is best recommended, RGB and XYZ have best speed while Lab94 and Lab00 have releatively good effect but are slow in speed. The HSV algorithm hasn't been working well for a long time, so it's not recommended. \textbf{Besides, AiCvter is the slowest but it's worth trying if all traditional algorithms fail to make you satisfied.}

    For customizing AiCvter, see \ref{CustomizeAiCvter}.
    
    Dithering uses Floyd-Steinberg algorithm, it tries to fit the orignal image better through mixing several similar colors.
    
    Here we take picture drawn by \href{https://t.bilibili.com/544583492149793294}{Lancet\_Corgi} as an instance(Thanks to Lancet\_Corgi, here's \href{https://space.bilibili.com/37171000}{his homepage on BiliBili})
    
    \begin{figure}[htbp]
        %\addtocounter{figure}{-1}
        \centering
        \setcounter{subfigure}{0}
        \subfigure[Before Convertion]{
            \begin{minipage}[t]{7cm}
                \centering
                \includegraphics[width=6cm]{Img3_Raw.png}                
                \label{ruby_raw}
            \end{minipage}
        }
        \subfigure[After Convertion]{
            \begin{minipage}[t]{7cm}
                \centering
                \includegraphics[width=6cm]{Img4_Converted.png}
                %\caption{转化后}
                \label{ruby_converted}
            \end{minipage}
        }
        \caption{Convert using RGB+(dithering disabled)}
    \end{figure}
    
    \subsection{Export 3D structure}
    If you hope to save the map in \textbf{.Litematic}(for Litematica mod) or \textbf{.nbt}(for Minecraft strcuture block) format, you should click \textbf{Build 3D} and then \textbf{Export}. When finished building, SlopeCraft will send you a preview window where you can see the material list and map.

    \begin{figure}[htbp]
        \centering
        \includegraphics[width=15cm]{Img2_Export3D.png}
        \caption{Interface of Exporing 3D}
    \end{figure}

    In this interface, \textbf{Intelligent Lossy Compress} and \textbf{Construct Glass Bridge} are added. And some other options like \textbf{Fire Proof} and \textbf{Enderman Proof} are added. A new sub progress bar is used to show the progress of lossy compression and construction of glass bridge.
        
    \subsubsection{Compression}
    Simply put, lossless compression is to compress the total height of 3D map at the cost of consecutiveness while \textbf{keeping colors of each pixel strictly unchanged}. It's a really rigorous restriction so it may fail to reach your expectation. It's no exaggeration to say that some images are mathematically uncompressable, like full-white parts. Then new compress technology is required: intelligent lossy compression.

    Different with the lossless one, lossy compression compress by \textbf{slightly modifying some pixel}, to make the total height no greater than the maximum allowed height(MAH) which is assigned by you. Genetic algorithm, which belongs to swarm intelligence, is employed to implement lossy compression. So you can see it's a speacial part of SlopeCraft with most technical content. The MAH should be no less than 14, otherwise your 3D map still have some probability of failing to be compressed.

    In general, the less MAH is, the more quality loss in your map. For example, is this map is made into a 3D map, its height will be 255. Now we compress it with both lossy and lossless algorithm. Firgue \ref*{ruby_max100} shows the result when MAH=100, and \ref*{ruby_max20} shows when MAH=20.

    \begin{figure}[htbp]
        \centering
        \subfigure[MAH=100]{
            \begin{minipage}[t]{7cm}            
                \centering
                \includegraphics[width=6cm]{Img6_Compressed_100.png}
                \label{ruby_max100}
            \end{minipage}
        }
        \subfigure[MAH=20]{
            \begin{minipage}[t]{7cm}
                \centering
                \includegraphics[width=6cm]{Img5_Compressed_20.png}
                \label{ruby_max20}
            \end{minipage}
        }
        \caption{MAH's effect to picture quality}
    \end{figure}

    There's no obvious difference before and after the compression. But you can still find that some gray dots appear on the left and right side. You can learn that the number of gray dots has a negative relation with MAH. Otherwise, as genetic algorithm is a stochastic optimizing algorithm, there's some randomness in which pixel will be modified and regular pattern won't appear.

    Lossy and lossless compression can be used either together or respectively. But in general, if you have already enabled lossy one, there's no way to disable the lossless one. Lossy compression without lossless will make a lot more pixels modified, bringing more quality loss, while the lossless counterpart make the compression be finished with less pixel modified and less quality loss.
    
    You can choose these options with flat map or wall map, but they \textbf{won't take any result}.

    \subsubsection{Construct the Glass Bridge}
    Each horizontal slice 3D map consists of seperated blocks, which is really hard to build. Connecting these blocks will obviously make a map easier to be built. Glassbridge is what we used to connect these blocks to form a pathway to assist players.
    
    It's definatly that the glass bridge will lead to more glass consumption, so it's deprecated to construct it in each layer of 3D map. In default, SlopeCraft constructs once in each 5 layers, and that interval can be modified. A too large interval will decrease the effect while a too small interval is a waste of glass.
    
    For any details about compression and glass bridge, read \href{https://github.com/ToKiNoBug/SlopeCraftTutorial/blob/main/BasicPrinciple/Principle%20of%20map%20pixel%20arts.md}{Principle of Map Arts}.

    \href{https://github.com/AbrasiveBoar902}{AbrasiveBoar902} made great contributions to the performance of constructing glass bridge, sincere thanks to AbrasiveBoar902.

    \subsubsection{Fire Proof / Enderman Proof}
    We can learn by name that these functions is added to protect burnable blocks and prevent enderman from stealing blocks in map. The detailed method is to conver every exposed surfaces of these blocks with glass blocks, and it does work. On the other hand it consumes a large amount of glass blocks.

    \subsubsection{Export}
    There are 2 formats supported currently, \textbf{*.Litematic} and \textbf{*.nbt}.

    \begin{figure}[htbp]
        \centering
        \includegraphics[width=15cm]{Img7_SelectFormat.png}
        \caption{Set Export Format}
        \label{setExport3DFormat}
    \end{figure}

    If you hope to save as \textbf{*.nbt} format, it's required to select the correct suffix, like figure \ref*{setExport3DFormat}.

    \subsection{Export as File-Only Map}
    File-only maps have few change compared to previous versions.

    File name of a map data file is like map\_i.dat, where i is an integer no less than 0, like map\_3.dat. \textbf{i is the sequence number of this map data file. Sequence number is the unqiue label of a map data file in Minecraft.} In normal conditons, your exported map data files shouldn't cover any unrelated counterparts, so be careful when setting the beginning sequence number.

    In Minecraft, pressing F3+h will enable you to learn details about an item, including map id, which is the sequence number of a map data file. Figure \ref*{mapItem} shows a map item of map\_6.dat.
   \begin{figure}[htbp]
       \centering
       \includegraphics[height=4cm]{Img8_MapItem.png}
       \caption{Map item and map sequence number}
       \label{mapItem}
   \end{figure}

   \begin{itemize}
       \item Get map using /give command:
       
       The beginning sequence number can be set arbitarary as soon as no unreleated map will be covered.
       \begin{enumerate}
           \item 1.12 command: /give @s filled\_map 1 i
           \item 1.13+ command: /give @s filled\_map\{map:i\}
       \end{enumerate}
       \item If you want to simply replace map data files instead of using commands:
       \begin{enumerate}
           \item Create n map items correspond to your image where n is the map count, it's 4 in this instance.
           \item Press F3+H to check the map sequence number of map items you've just created, their sequence numbers should be a\textasciitilde(a+n-1), n in total.
           \item Close Minecraft and fill the value of a in the \textbf{Beginning Sequence Number} branket.
           \item Click export and then select data folder under you save. SlopeCraft will replace these map data files. 
           \item Close SlopeCraft and start Minecraft, these map items will be replaced with map pixel arts successfully.
           \item If you are afraid of your misoperation which may lead to unreleated map being covered, you can create a temporary folder and then selected when exporting. Repleace the map data files you want to replace after you're sure that no error occurres.
       \end{enumerate}
   \end{itemize}
   
   \pagebreak
   \section{Advanced Functions}
   Basical function is enough if you just want to use SlopeCraft simply. However, if you want to customize something, you'd better have a look at this section. Using advanced functions usually requires understanding of map arts' principles. It's strongly recommended to read \href{https://github.com/ToKiNoBug/SlopeCraftTutorial/blob/main/BasicPrinciple/Principle%20of%20map%20pixel%20arts.md}{Principle of map pixel arts}.

   \subsection{Batch operation}
   \label{BatchOp}
   This function aims to make multiple images into map data files or schematics.

   It's really easy to start, just select multiple image files when importing image. But before doing so, you must have all settings done.

   You can jump to most pages via the left sidebar. So set the type of your maps, the blocklist, the convertion algorithm, and all exporting parameters. 
   
   Mention that you can't jump directly to the two exporting page before a image is converted, but there's a lane: click \textit{Advanced} and \textit{Batch operation} and then \textit{Set exporting parameters} and you will jump to the page of exporting schematics before convertion. There's no skipping access to the page of exporting map data files since they have fewer parameters than schematics.

   After assigning all parameters, select multiple images and a window will pop out like figure \ref*{BatchUi}. You can select the exporting format (litematica schematics, vanilla structures and map data files). If map data files are to be generated, you must choose a folder to put then and set their beginning sequece numbers while schematics' filenames and directories aren't editable now.

   Click \textit{Start} and all images will be finished automatically.

   \begin{figure}[htbp]
       \centering
       \includegraphics[width=15cm]{Img12_BatchOp.png}
       \caption{Interface of batch operation}
       \label{BatchUi}
   \end{figure}
   
   \subsection{Blocklist presets}
   A blocklist preset is a preset that assigned whether each basecolor is used and it's block id to be used. It is stored as json with \textit{*.scPreset.json} suffix.

   Find \textit{Blocklist preset} submenu below \textit{Advanced} menu, where you can save current blocklist as preset files, or load from a preset file.

   \subsection{Customize blocklist}
   If you aren't satisfyed with default blocks, which means that you hope to add other vanilla blocks or mod-blocks, this section will tell you what to do.
   
   \subsubsection{Pre-informations}
   You must master following information first:

   \begin{enumerate}
       \item \textbf{Complete} id of this block, including \textbf{namespace prefix} and \textbf{all block properties}.
       
       For instance, waxed copper slab:
       
       minecraft:waxed\_copper\_slab[type=top,waterlogged=false]

       Here minecraft: is the namespace prefix of vanilla blocks, contents in the brackets are all block properties. For safety reason, you should set a value for every block properties.

       \item The earliest Minecraft version when this block is added.

       SlopeCraft assigned the following numbers as references of several Minecraft versions:
       \begin{table}[h]
        \centering
        \caption{Numbers and versions}
        \label{VerAndRealVer}
        \begin{tabular}{cc}\hline
            Number & Version \\ \hline
            0 & Earlier than 1.12 \\
            12 & 1.12 \\
            13 & 1.13 \\
            14 & 1.14 \\
            15 & 1.15 \\
            16 & 1.16 \\
            17 & 1.17 \\
            255 & Future version \\
            \hline            
        \end{tabular}
       \end{table}

   Usually you shouldn't use 255, it's just a reserved value. If you insist to do so, then everyting will be undefined features ---- I don't know what will happen.

   \item Block id in 1.12.
   
   This property is added simply because a great number of block ids changed from 1.12 to 1.13. If your block doesn't exist in 1.12, or its id kept unchanged, you can fill in an empty string.
   
   \item Base color of this block.
   
   Perhaps the most error prone property, but also the extremely and most important one. For vanilla blocks, view \href{https://minecraft.fandom.com/wiki/Map_item_format}{Minecraft Wiki}. For mod-blocks, measure the basecolor by yourself or ask the mod developer. Asking me will never take effort.
   
   If you don't understand what basecolor is, go to read \href{https://github.com/ToKiNoBug/SlopeCraftTutorial/blob/main/BasicPrinciple/Principle%20of%20map%20pixel%20arts.md}{Principle of map pixel arts}.

   \item Chinese name of block (arbitrary value if you don't speak Chinese).
   \item English name of block
   \item Whether another block is required under this block
   \item Whether the block glows
   \item Whether the block is burnable
   \item Whether the block can be stolen by enderman.
   \item Whether the block is suitable for wall maps(for instance, water and iron pressure plate is not suitable)
   \item Texture image of this block (16*16px png is suggested)  
   \end{enumerate}

   \subsubsection{Blocks and Blocklists}
   SlopeCraft stores blocklists in json format, while corresponding images are stored in FixedBlocks and CustomBlocks directories.
   
   There are two blocklists: fixed blocklist and cutsom blocklist. The former provides fundemental blocks which are stored in \textbf{FixedBlocks.json}(corresponding images in \textbf{FixedBlocks} directory), ensuring that each basecolor has at least one block. Although blocklists are determined at runtime, you \textbf{shouldn't modify the fixed blocklist}.
   
   The other blocklist is stored in \textbf{CustomBlocks.json}(corresponding images in \textbf{CustomBlocks} directory), this is where you add your custom blocks. Some slabs have been already written for instances.
   
   Each block has the following properties:
   \begin{table}[h]
    \centering
    \caption{Block properties}
    \begin{tabular}{ccccc}
        \hline
        Property & Type & Is compulsory & Default value & Notes  \\ \hline
        baseColor & byte & Yes & & Basecolor of the block \\
        id & string & Yes & & Complete block id\\
        version & byte & Yes & & Earliest version of the block \\
        nameZH & string & Yes & & Chinese name of the block \\
        nameEN & string & Yes & & English name of the block \\
        icon & string & Yes & & Filename of related image file \\
        idOld & string & No & Empty string & Block id in 1.12 \\
        needGlass & bool & No & false & If the block needs a block below \\
        isGlowing & bool & No & false & It the block emits light \\
        endermanPickable & bool & No & false & If can be picked by enderman \\
        burnable & bool & No & false & If the block is burnable \\
        wallUseable & bool & No & true & If the block is suitable for wall map \\
        \hline 
    \end{tabular}       
   \end{table}
   
   \clearpage
   Json code:
\begin{lstlisting}[language = C++, numbers=left, 
    numberstyle=\tiny,keywordstyle=\color{blue!70},
    commentstyle=\color{red!50!green!50!blue!50},frame=shadowbox,
    rulesepcolor=\color{red!20!green!20!blue!20},basicstyle=\ttfamily]
{
    "baseColor":11,
    "id":"minecraft:cobblestone_slab[type=top,waterlogged=false]",
    "nameZH":"圆石上半砖",
    "nameEN":"Cobblestone slab",
    "icon":"cobblestone.png",
    "version":0,
    "idOld":"minecraft:stone_slab[half=top,variant=cobblestone]"
}
    \end{lstlisting}
    The json code above shows block properties of a upper cobbleston slab, here's the interpretation:

    \begin{enumerate}
        \item It's basecolor is 11, same as stone block and cobblestone.
    
        \item It's block id is “minecraft:cobblestone\_slab[type=top,waterlogged=false]”, block statues in the brackets indicate that it's a upper slab and is not waterlogged.
    
        \item It's Chinese name is damaged due to the fucking encoding, and its English name is "Cobblestone slab".
    
        \item It's releated image is a png file named “cobblestone.png”, which is stored in CustomBlocks directory.
    
        \item It's earliest version is 0, which means that it's added before 1.12.
    
        \item It's block id has been changed in 1.13, so idOld shows it's old id in 1.12.

\end{enumerate}

    The json above doesn't show everything about this block, properties named needGlass, isGlowing, endermanPickable, burnable and wallUseable are set to default values. Thus, this block doesn't need to adhere to another block below, and it doesn't emit light, and enderman can't pick it, and it doesn't burn, and it is suitable for wall maps.

    \subsubsection{Do It Yourself}
    Having master informations as above, you are able to write in any block you want. Steps:

    \begin{enumerate}
        \item Make the your block's image into a 16 by 16 px png and move it into CustomBlocks directory.
        \item Fill in your block's json information into CustomBlocks.json, don't make mistake in json format!
        \item Restart SlopeCraft. If nothing goes wrong ,your block shall be added into blocklist successfully, otherwise remind the error message.
        \item Have fun to make your map pixel arts with SlopeCraft.
    \end{enumerate}

    \subsection{Test Your Blocklist}
    Most block-missing errors are caused by \textbf{WROUNG ID SPELLING}. If you imported lots of blocks, this function can help you to find and correct id mistakes.

    Testing blocklist will generate a special structure file, containing every blocks owned by every basecolors(unless the block is not useable for version reason). In the structure file, blocks are mapped in the same order as they are shown in the blocklist interface.
    
    If everything goes right, no block will be missing; otherwise it means an id mistake.
    
    After you have the game version selected, find and click \textbf{Test block list} under \textbf{Advanced}, then choose a directory to put the structure file. SlopeCraft will then generate a structure like figure \ref*{testBlockListNBT}.

    \begin{figure}[htbp]
        \centering
        \subfigure[Left part]{
            \begin{minipage}[t]{7cm}
                \centering
                \includegraphics[width=6.5cm]{Img10_TestBlockList_Left.png}
            \end{minipage}
        }
        \subfigure[Right part]{
            \begin{minipage}[t]{7cm}
                \centering
                \includegraphics[width=6.5cm]{Img11_TestBlockList_Right.png}
            \end{minipage}
        }
        \caption{Test bloclist}
        \label{testBlockListNBT}
    \end{figure}

    \subsection{Customize the AiCvter}
    \label{CustomizeAiCvter}
    The AiCvter uses genetic algorithm(GA), you can change its parameters in submenu \textit{Ai converter parameters} under menu \textit{Advanced}.

    \begin{figure}[htbp]
        \centering
        \includegraphics[width=10cm]{Img13_AiCvterPara.png}
        \label{GAOpt}
        \caption{AiConverter parameters}
    \end{figure}

    For those who understand GA, there's no need to explain these parameters. \textbf{If you don't know about GA, google it, otherwise please don't change them.}

    The only one I need to introduce is \textit{maxFailTimes}. It means the maximum allowed continuous generations that GA failed to find a better solution. For example, if it's set to 50 and a GA solver hasn't been finding a better solution for 50 generations, the algorithm will stop to save time. A greater \textit{maxFailTimes} will reduce the probability of being premature but will make it less faster.

    
\pagebreak
\section{Next update}
    The next version(v3.8) will have following new functions \st{probabaly}:
    \begin{enumerate}
        \item Refining converted map manually (well, stood up in this update)
        \item Material restriction when building 3D structure
        \item ......
    \end{enumerate}

\end{document}
