%\documentclass[UTF8]{ctexart}
\documentclass{article}
\usepackage{geometry}
\usepackage{fancyhdr}
\usepackage{verbatim}
\usepackage{enumerate}
\usepackage{graphicx}
\usepackage{subfigure}
\usepackage[colorlinks,linkcolor=blue]{hyperref}
\usepackage{listings}
\usepackage{fontspec}
\setmonofont{Consolas}
\usepackage{color}
\usepackage{xcolor}
\graphicspath{{./Images/}}

\geometry{papersize={210mm,297mm}}
\geometry{left=2.5cm,right=2.5cm,top=2.5cm,bottom=2.5cm}

\setlength{\headheight}{13pt}

\title{SlopeCraft v3.6 UserGuide}
\author{TokiNoBug}
\date{\today}

\begin{comment}
\pagestyle{fancy}

\lhead{\author}
\chead{\title}
\rhead{\date}

\lfoot{}
\cfoot{\thepage}
\rfoot{}

\renewcommand{\headrulewidth}{0.4pt}
\renewcommand{\headwidth}{\textwidth}
\renewcommand{\footrulewidth}{0pt}

\end{comment}

\begin{document}
    \maketitle
    %%%%%%%%%%%%%%%%%%%%%%%
    % Page1
    %%%%%%%%%%%%%%%%%%%%%%%
    SlopeCraft is such a great update that the whole software is rewritten again with Many hardcore functions implemented and lots of bugs fixed. In this version, lossy compression is added to compress total height of a 3D map to almost any arbitrary value, which entirely sloves the problem that 3D map often got higher than the world. Also, glassbridge is added in order to assitant player when building.
    
    Besides, ui is slightly changed. Since the last tutorial is for v3.1, it's necessary to write a new user guide.

    By the way, this tutorial is not a fool-type one, only those must be introduced are introduced and those must be underlined are underlined. If any question, view a previous tutorial.

    %%%%%%%%%%%%%%%%%%%%%%%
    % Page2
    %%%%%%%%%%%%%%%%%%%%%%%
    \pagebreak
    \section{Update Summary}
    \paragraph{New Contents}

    \textbf{SlopeCraftL3.dll}——The kernel of SlopeCraft is developed into a dynamic linked library through a interface class, including core functions of 3d, flat, file-only and wall maps. Theoretically, this lib can be called by not only C++ but also Java and Python and so on.(When it comes to other languages, one more layer of interface might be required). Future plugins will have one more choice.

    \paragraph{New Functions}
    \begin{enumerate}
        \item The 61th basecolor
        \item Intelligent lossy compression
        \item Glassbridge construction——\textbf{Easy to build 3d maps}
        \item Fireproof and enderman-proof——cover vulnerable blocks with glass.
        \item Export as .NBT format——Minecraft structure block format.
        \item All kinds of \textbf{Upper slabs} to blocklist.
        \item Customizable blocklist——add any block if you like.
        \item Preview map when constructing 3d structure——\textbf{Lossy compression may edit it slightly}
        \item Check for updates and report bugs——by subject to \href{https://github.com/ToKiNoBug/SlopeCraft}{repository on GitHub}.
        \item More complete error-reporing function.
        \item Reset language automatically when starting——if Chinese isn't found in system language, English instead.
        \item settings.json——config files for starting.
        \item New map type: wall maps——kind of weird but requested for many times.
    \end{enumerate}
    \paragraph{Bug Fixed}
    \begin{enumerate}
        \item Fixed the bug that may colors can't be disabled.
    \end{enumerate}
    \paragraph{Optimization and Other Changes}
    \begin{enumerate}        
        \item Optimized the performance of converting and constructing glassbridge.
        \item Deleted unnecessary \textbf{Confirm} buttons. Less opeartions required now.
        \item Progessbars are no longer in busy state when waitting for user's operation. More suited for users now.
    \end{enumerate}

    \pagebreak
    %%%%%%%%%%%%%%%%%%%%%%%
    % Page3
    %%%%%%%%%%%%%%%%%%%%%%%
    \section{Junior Tutorial}
    \subsection{Import Image}
    From the previous tutorial to now, the greatest change is still a new button \textbf{Settings}, which is used to process transparent pixels better.
 
    Click \textbf{Settings} and you will see a subwindow as figure \ref*{SetTPS}.

    \begin{figure}[htbp]
        \centering
        \includegraphics[width=15cm]{Img1_TPS.png}
        \caption{Subwindow to set transparence processing strategy}
        \label{SetTPS}
    \end{figure}
    
    \textbf{Note: If you want to import a image with transparency under custom transparency processing strategy, you must set the strategy first and then import you image! Otherwise the image will only be preprocessed by the default strategy! If you set the strategy after having your image imported, import you image again.}

    透明像素处理策略有不同的方法处理纯透明像素(alpha=0)和半透明像素(alpha>1)。纯透明像素要么替换为背景色,要么设为空气;半透明像素既可以替换为背景色,也可以与背景色叠加融合,还可以忽略掉它的透明度属性,直接当做不透明像素。另外也可以设置背景色,默认背景色是雪块平铺时的浅灰色,也可以选择纯白色,或者任何自定义颜色。

    强调一下,原图中\textbf{每个像素对应着地图画的一个方块},导入图片之前需要自己裁剪、缩放图像,图片的长和宽最好是\textbf{128像素的整数倍}(不是强制要求)。这一步预处理应当由用户自己完成,我不在软件中实现这个功能,也不存在任何缩放图像的操作,SlopeCraft忠实的映射\textbf{每个像素}。

    \subsection{设置地图画类型}
    地图画类型中多了一项墙面像素画,除此以外没有改动。只需要按说明设置地图画类型及版本,然后直接跳转到下一页即可。

    设置方块列表页面补加了第61个基色:发光苔藓。这是1.17已经添加的基色,但v3.5.1漏了。
    请注意方块列表的逻辑:灰色的按钮/选择框代表锁定、不可更改的,如基色0(透明)必须启用。SlopeCraft要求每种基色必须有对应的方块,哪怕这种基色本身被禁用。因此如果某个基色只有一种方块可选,那个唯一的单选框也是锁定的。
    
    如果一种方块版本过高,它同样会变为灰色,不可选择。有必要指出,有些方块是墙面像素画不能使用的,如铁质压力板、发光苔藓等。这时限制颜色数量的不是基色/地图色本身的性质,而是你所选的方块。

    \subsection{调整颜色}
    调整颜色界面也没有值得一提的改变。
    
    首先选择好转化算法,然后选择是否启用抖动,点击\textbf{转化},等待转化完成,就可以选择你想要的导出方式。
    
    六个转化分别对应着六种不同的色差公式。算法中RGB+最为推荐,RGB和XYZ速度最快,Lab94和Lab00效果较好但性能较慢,HSV效果一直不太理想,不太推荐。
    
    抖动则使用Floyd-Steinberg算法,尝试用几种将近的颜色掺混,试图更好的贴合原图。
    
    这里以\href{https://t.bilibili.com/544583492149793294}{Lancet\_Corgi画的图}为例。(\href{https://space.bilibili.com/37171000}{Lancet\_Corgi的b站主页},感谢\~)
    
    \begin{figure}[htbp]
        %\addtocounter{figure}{-1}
        \centering
        \setcounter{subfigure}{0}
        \subfigure[原图]{
            \begin{minipage}[t]{7cm}
                \centering
                \includegraphics[width=6cm]{Img3_Raw.png}
                %\caption{原图}
                
                \label{ruby_raw}
            \end{minipage}
        }
        \subfigure[转化后]{
            \begin{minipage}[t]{7cm}
                \centering
                \includegraphics[width=6cm]{Img4_Converted.png}
                %\caption{转化后}
                \label{ruby_converted}
            \end{minipage}
        }
        \caption{导入原图并按RGB+算法(无抖动)转化}
    \end{figure}
    
    \subsection{导出三维结构}
    若想要把地图画保存为\textbf{.Litematic}(投影mod)、\textbf{.nbt}(结构方块)格式,须先点击\textbf{构建三维结构},然后再\textbf{导出}。构建三维结构完成之后会自动弹出预览窗口,可以查看材料表,也可以查看地图画。

    \begin{figure}[htbp]
        \centering
        \includegraphics[width=15cm]{Img2_Export3D.png}
        \caption{导出三维结构界面}
    \end{figure}

    压缩高度一栏增加了\textbf{智能有损压缩};增加了\textbf{搭桥}选项;增加了\textbf{防火}和\textbf{防末影人}选项。 在主进度条的右侧多了一个子进度条,表示有损压缩/搭桥的进度。
        
    \subsubsection{压缩}
    简单来说,无损压缩是在\textbf{严格保证每个像素颜色不变}的前提下,以连续性为代价,压缩地图画总高度,但它约束较大,未必如愿。毫不夸张的说,有些图片是不可压缩的,比如纯白色的部分。这时候就需要新的压缩技术:智能有损压缩。

    有损压缩则\textbf{微调个别像素的颜色},压缩地图画总高度,使其小于等于用户指定的最大允许高度。有损压缩使用遗传算法实现,属于群体人工智能,是目前SlopeCraft中技术含量最高的模块。最大允许高度不要低于14,否则立体地图画很可能压缩失败。

    一般来说,有损压缩的最大允许高度越小,画质损失越明显。如下,这张图若做成立体地图画,高度是255格,现在进行有损压缩(同时启用无损压缩)。图\ref*{ruby_max100}为最大高度100格时的构建结果,图\ref*{ruby_max20}为最大高度20格时的构建结果。

    \begin{figure}[htbp]
        \centering
        \subfigure[最大100格]{
            \begin{minipage}[t]{7cm}            
                \centering
                \includegraphics[width=6cm]{Img6_Compressed_100.png}
                \label{ruby_max100}
            \end{minipage}
        }
        \subfigure[最大20格]{
            \begin{minipage}[t]{7cm}
                \centering
                \includegraphics[width=6cm]{Img5_Compressed_20.png}
                \label{ruby_max20}
            \end{minipage}
        }
        \caption{最大允许高度对画质的影响}
    \end{figure}

    压缩前后图片没有显著变化,画质损伤不明显。但仔细观测仍可以发现,左右两侧留白部分出现了一些灰点,且图\ref*{ruby_max20}由于压缩程度高,灰点较图\ref*{ruby_max100}更多。另外,遗传算法是一种随机优化算法,被修改像素有一定随机性,不会呈现明显的规律图样。

    有损压缩和无损压缩可以搭配使用,也可以分别独立使用。但一般来说,如果启用了有损压缩,没道理不启用无损压缩。纯有损压缩需要修改更多的像素,对画质的损伤会比较大,无损压缩能在修改像素更少的情况下完成压缩任务,很大程度上减轻画质损伤。
    
    平板地图画和墙面像素画可以勾选这两个选项,但\textbf{不会发挥任何作用}。

    \subsubsection{搭桥}
    立体地图画的每个水平截面上都有很多分散的方块,极不方便建造,倘若能用多条通路连接这些分散的方块,无疑能让建造更加容易。搭桥就是在一个水平面内用玻璃方块连接所有方块,形成通路从而辅助玩家建造的过程。
    
    毫无疑问,搭桥会消耗额外的玻璃,因此不推荐在立体地图画的每一层都执行搭桥。默认每间隔4层搭一次桥,你也可以修改这个间隔。间隔过大,辅助搭桥的效果会减弱;间隔过小,浪费玻璃。
    
    有关地图画压缩和搭桥的详细信息,可阅读\href{https://github.com/ToKiNoBug/SlopeCraftTutorial/blob/main/BasicPrinciple/Principle%20of%20map%20pixel%20arts.md}{地图画原理}。

    \href{https://github.com/AbrasiveBoar902}{AbrasiveBoar902}为优化搭桥性能提供了极大帮助,感谢AbrasiveBoar902的帮助。

    \subsubsection{防火/防末影人}
    顾名思义,这是在保护可燃方块,并避免小黑偷东西。具体方法是用玻璃包裹这些方块的每一个暴露在外的表面,亲测有效。不过这也同时会耗费大量的玻璃,需要谨慎选择。

    \subsubsection{导出}
    目前支持的有\textbf{*.Litematic}(投影mod)和\textbf{*.nbt}(原版结构方块)两种格式。

    \begin{figure}[htbp]
        \centering
        \includegraphics[width=15cm]{Img7_SelectFormat.png}
        \caption{设置导出格式}
        \label{setExport3DFormat}
    \end{figure}

    若要保存为结构方块格式,须在导出时选择相应的文件后缀名,如图\ref*{setExport3DFormat}所示。

    \subsection{导出纯文件地图画}
    和以前的版本相比,导出为纯文件并没有什么变化。

    地图文件的文件名形如map\_i.dat,其中i是大于等于0的整数,如map\_3.dat。\textbf{i就是这个地图文件的序号。序号实际上是地图文件的唯一标识符}。正常情况下,我们生成的地图文件不应该覆盖掉无关的地图文件,所以设置初始序号需多加注意。

   在游戏中按下F3+H可以查看物品的详细信息,\textbf{包括地图物品的Id,也就是对应的地图文件的序号}。图\ref*{mapItem}中显示的地图物品对应名为map\_6.dat的地图文件。
   \begin{figure}[htbp]
       \centering
       \includegraphics[height=4cm]{Img8_MapItem.png}
       \caption{地图物品与地图序号}
       \label{mapItem}
   \end{figure}

   \begin{itemize}
       \item 如果你想通过/give命令来获得地图:
       
       起始序号可以随意设置,只要不覆盖掉无关的地图。
       \begin{enumerate}
           \item 在1.12,使用 /give @s filled\_map 1 i 来获得序号为i的地图。
           \item 在1.13+,使用 /give @s filled\_map\{map:i\} 来获得序号为i的地图。
       \end{enumerate}
       \item 如果你不想使用命令,只替换地图文件:
       \begin{enumerate}
           \item 先创建与地图画对应的n个地图,n就是SlopeCraft显示的地图文件数量,在本例中是4。
           \item 在游戏中按下F3+H,查看地图文件对应的序号。这些地图对应的序号应当是 a\textasciitilde(a+n-1) ,共n个。
           \item 关闭游戏,在SlopeCraft的\textbf{地图文件起始序号栏}中填入a的值。
           \item 点击导出,选中存档下的data文件夹。SlopeCraft将会替换掉这n个地图文件。
           \item 关闭SlopeCraft,打开游戏,这n个地图应当已经被成功的替换为地图画。
           \item 如果你担心输错地图文件序号,导致无关的地图被覆盖掉,你可以先新建一个临时的文件夹,在导出时选择这个临时文件夹。确认地图序号无误后,再复制黏贴替换掉你要替换的地图文件。
       \end{enumerate}
   \end{itemize}
   
   \pagebreak
   \section{高级功能}
   如果你只是简单的使用SlopeCraft,掌握初级功能足矣;但如果你需要自定义一些东西,那么你最好阅读下这一节。使用高级功能往往需要理解地图画的基本原理,我强烈建议你先看完\href{https://github.com/ToKiNoBug/SlopeCraftTutorial/blob/main/BasicPrinciple/Principle%20of%20map%20pixel%20arts.md}{地图画原理}。

   \subsection{自定义方块列表}
   如果你不满足于我预设的那些方块,想要自己添加其他的原版方块甚至mod方块,这一章会告诉你怎么在SlopeCraft中添加并使用自定义的方块。
   
   \subsubsection{前置信息}
   你需要掌握方块的以下信息:

   \begin{enumerate}
       \item 方块的\textbf{完整}id,包含\textbf{命名空间前缀}以及\textbf{所有方块属性}。
       
       如涂蜡铜块上半砖:
       
       minecraft:waxed\_copper\_slab[type=top,waterlogged=false]

       这里面minecraft:是原版方块的命名空间前缀,中括号里的内容是所有方块属性。保险起见,你应当给每个方块属性都设置对应的值。
       \item 方块最早出现的游戏版本。
       
       SlopeCraft在方块列表中约定了以下几个值代指大版本:
       \begin{table}[h]
        \centering
        \caption{数字与版本的关系}
        \label{VerAndRealVer}
        \begin{tabular}{cc}\hline
            数字 & 版本 \\ \hline
            0 & 早于1.12 \\
            12 & 1.12 \\
            13 & 1.13 \\
            14 & 1.14 \\
            15 & 1.15 \\
            16 & 1.16 \\
            17 & 1.17 \\
            255 & 未来版本 \\
            \hline            
        \end{tabular}
       \end{table}

   正常情况下,你不应该使用255,它只是一个预留的值。如果你非要给一个方块指定为未来版本,那么导致的一切都属于未定义特性——我也不知道会发生什么。

   \item 方块在1.12的id
   
   添加这个属性是因为Mojang从1.12更新到1.13修改了相当多方块的id。如果你要添加的方块在1.12未添加,或者id没有改变,可以填空字符串。
   \item 方块的基色
   
   这可能是最容易出错的地方。对于原版方块,你可以查询\href{https://wiki.biligame.com/mc/%E5%9C%B0%E5%9B%BE%E7%89%A9%E5%93%81%E6%A0%BC%E5%BC%8F#idcounts.dat_.E6.A0.BC.E5.BC.8F}{Minecraft Wiki}。如果是mod自定义的方块,要么自己想办法测,要么去问mod开发者。
   
   如果不懂什么是基色,去看\href{https://github.com/ToKiNoBug/SlopeCraftTutorial/blob/main/BasicPrinciple/Principle%20of%20map%20pixel%20arts.md}{地图画原理}。
   \item 方块中文名称
   \item 方块英文名称
   \item 方块的下方需要依附其他方块
   \item 方块是否发光
   \item 方块是否可燃
   \item 方块是否可被末影人偷走
   \item 方块是否可用于墙面像素画(如水、铁质压力板等方块是不适合的)
   \item 方块的材质图片(建议为一张16*16像素的png图片)   
   \end{enumerate}

   \subsubsection{方块与方块列表}
   SlopeCraft中,方块列表以json格式存储,相关的图片放在FixedBlocks和CustomBlocks文件夹下。
   
   方块列表分为两类:固定方块列表与自定义方块列表。其中固定方块列表是我提供的最基础的一些方块,存储与\textbf{FixedBlocks.json}中(对应的图片在\textbf{FixedBlocks}文件夹下),确保每个基色都存在对应的方块。虽然方块列表中的方块事实上是在程序运行时才确定,但\textbf{你不应该修改它}。
   
   自定义方块列表存储在\textbf{CustomBlocks.json}中,相应的图片在\textbf{CustomBlocks}文件夹下,这是用户自定义的方块列表,允许用户灵活编辑。我在里面写了一些半砖方块,可以作为参考。
   
   每个方块拥有以下属性:
   \begin{table}[h]
    \centering
    \caption{方块的属性}
    \begin{tabular}{ccccc}
        \hline
        属性名 & 类型 & 是否必填 & 默认值 & 说明  \\ \hline
        baseColor & byte & 是 & & 方块的地图基色 \\
        id & string & 是 & & 方块id,包括命名空间和详细方块状态\\
        version & byte & 是 & & 方块最早出现的版本。 \\
        nameZH & string & 是 & & 方块的中文名 \\
        nameEN & string & 是 & & 方块的英文名 \\
        icon & string & 是 & & 方块对应图片的文件名 \\
        idOld & string & 否 & 空字符串 & 方块在1.12的id \\
        needGlass & bool & 否 & false & 指示方块底部是否必须有其他方块 \\
        isGlowing & bool & 否 & false & 指示方块是否发光 \\
        endermanPickable & bool & 否 & false & 指示方块是否可以被末影人偷走 \\
        burnable & bool & 否 & false & 指示方块是否可以被烧毁 \\
        wallUseable & bool & 否 & true & 指示方块是否可以用于墙面像素画 \\
        \hline 
    \end{tabular}       
   \end{table}
   
   \clearpage
   用json格式表示如下:
\begin{lstlisting}[language = C++, numbers=left, 
    numberstyle=\tiny,keywordstyle=\color{blue!70},
    commentstyle=\color{red!50!green!50!blue!50},frame=shadowbox,
    rulesepcolor=\color{red!20!green!20!blue!20},basicstyle=\ttfamily]
{
    "baseColor":11,
    "id":"minecraft:cobblestone_slab[type=top,waterlogged=false]",
    "nameZH":"圆石上半砖",
    "nameEN":"Cobblestone slab",
    "icon":"cobblestone.png",
    "version":0,
    "idOld":"minecraft:stone_slab[half=top,variant=cobblestone]"
}
    \end{lstlisting}
    上面这段json信息展示了一个圆石上半砖的方块信息,解析如下:

    \begin{enumerate}
        \item 它的基色是11,这也是圆石、石砖、石头的基色。
    
        \item 它的方块id是“minecraft:cobblestone\_slab[type=top,waterlogged=false]”,中括号里的方块状态指出这是个上半砖,且不含水;
    
        \item 它的中文名是“圆石上半砖”,英文名是“Cobblestone slab”;
    
        \item 它的图片是一个名为“cobblestone.png”的图片,这个图片放在CustomBlocks文件夹下;
    
        \item 它最早出现的版本为0,代表它在1.12之前就已经加入;
    
        \item 由于它的方块id在1.13发生了变化,它在1.12的方块id为idOld的值。

\end{enumerate}

    这段json中并没有显式列出方块的所有信息,其中needGlass、isGlowing、endermanPickable、burnable、wallUseable这些信息都使用了默认值,分别说明这个方块不需要依附其他方块、不发光、末影人偷不走、不可燃、可用于墙面像素画。

    \subsubsection{自己动手,丰衣足食}
    掌握以上信息后,就可以向自定义方块列表里写入任何你想要的方块了。步骤如下:

    \begin{enumerate}
        \item 将方块的图片做成16*16像素的png图片,放入CustomBlocks文件夹内。
        \item 将方块的json信息填入CustomBlocks.json中,注意json格式不要出错。
        \item 重启SlopeCraft,如果一切正常,你的方块将会成功加入到方块列表里;否则注意察看报错信息。
        \item 重复制作地图画的流程。
    \end{enumerate}

    \subsection{测试方块列表}
    投影中方块缺失基本上都是\textbf{id拼写错误}造成的。如果你一次性导入很多方块,这个功能可以快速测试出方块列表中每个方块有没有id错误。

    测试方块列表会生成一个特殊的结构方块文件,包含每种基色拥有的每一种可用方块(因版本不符合的除外)。结构文件中每个方块均按方块列表中的顺序排列。
    
    如果一切正常,不会有任何一个方块缺失;反之则说明对应的方块存在id错误。
    
    \begin{figure}[htbp]
        \centering
        \includegraphics[width=15cm]{Img9_TestBlockList.png}
        \caption{测试方块列表在高级菜单下}
        \label{locOfTestBlockList}
    \end{figure}

    设置好游戏版本后,点击菜单栏\textbf{高级}下的\textbf{测试方块列表},选择保存这个结构方块文件的位置,如图\ref*{locOfTestBlockList}。SlopeCraft就会生成一个这样的结构方块文件。导入到游戏中即可,效果如图\ref*{testBlockListNBT}。

    \begin{figure}[htbp]
        \centering
        \subfigure[左半段]{
            \begin{minipage}[t]{7cm}
                \centering
                \includegraphics[width=6.5cm]{Img10_TestBlockList_Left.png}
            \end{minipage}
        }
        \subfigure[右半段]{
            \begin{minipage}[t]{7cm}
                \centering
                \includegraphics[width=6.5cm]{Img11_TestBlockList_Right.png}
            \end{minipage}
        }
        \caption{测试方块列表}
        \label{testBlockListNBT}
    \end{figure}

%\pagebreak
\section{下个版本}
    不出意外,下个版本v3.7将会加入以下功能:
    \begin{enumerate}
        \item 跟进Minecraft1.18
        \item 批量生成地图画
        \item 手动精修
        \item ……
    \end{enumerate}

\end{document}
