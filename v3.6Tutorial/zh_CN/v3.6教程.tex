\documentclass[UTF8]{ctexart}
\usepackage{geometry}
\usepackage{fancyhdr}
\usepackage{verbatim}

\geometry{papersize={210mm,297mm}}
\geometry{left=2.5cm,right=2.5cm,top=2.5cm,bottom=2.5cm}

\setlength{\headheight}{13pt}

\title{SlopeCraft v3.6教程}
\author{TokiNoBug}
\date{\today}

\begin{comment}
\pagestyle{fancy}

\lhead{\author}
\chead{\title}
\rhead{\date}

\lfoot{}
\cfoot{\thepage}
\rfoot{}

\renewcommand{\headrulewidth}{0.4pt}
\renewcommand{\headwidth}{\textwidth}
\renewcommand{\footrulewidth}{0pt}

\end{comment}

\begin{document}
    \maketitle
    SlopeCraft v3.6是一个重大更新,整个程序都被重构了一遍,增加了许多硬核的功能,解决了不少棘手的麻烦:v3.6实现了期盼已久的有损压缩功能,可以把地图画总高度压缩到几乎任何值,彻底解决了地图画超出限高的问题;v3.6加入了“搭桥”功能,可以在每个水平面内,用玻璃方块连接所有方块,形成通路,辅助玩家建造……
    
    除此以外,v3.6还修改了界面的操作逻辑。加之上一版教程还是v3.1,我认为有必要写个新的教程。

    这篇教程不是“傻瓜式”的,我只介绍必须介绍的。所有\textbf{加粗}的内容都是一定要注意的、不注意就会出错的重点。

    \pagebreak
    page 2
\end{document}